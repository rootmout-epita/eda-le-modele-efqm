% Commandes customs pour faciliter et accélérer la rédaction du doc.

% Créé un bloc bleu de définition.
\newcommand{\definitionbloc}[1]{
    \begin{mybox}{custom_blue}
        \textbf{\textcolor{custom_blue}{Définition}  \Gls{#1}}
        \vspace{2mm} \\
        \glsdesc*{#1}
    \end{mybox}
}

\newcommand{\definitionblocen}[1]{
    \begin{mybox}{custom_blue}
        \textbf{\textcolor{custom_blue}{Definition}  \Gls{#1}}
        \vspace{2mm} \\
        \glsdesc*{#1}
    \end{mybox}
}

% Créé un bloc gris de note.
\newcommand{\notebloc}[1]{
    \begin{mybox}{custom_gray}
        \textbf{\textcolor{custom_gray}{Note}}
        \vspace{2mm} \\
        {#1}
    \end{mybox}
}

% Ajoute une entrée dans la glossaire en citant la source.
\newcommand{\newdefinition}[4]
{
    \newglossaryentry{#1}
    {
        name={#2},
        description={{#4}\vspace{1mm}\\
        {\small\mycitation{#3}}}
    }
}

% Cite un document sous un format personnalisé.
\newcommand{\mycitation}[1]{
    \textit{\citefield{#1}{title} \cite{#1}}
}

% Cite une figure sous un format personnalisé.
\newcommand{\myfigurereference}[1]
{(\textit{voir fig.\ref{#1} page \pageref{#1}})}
