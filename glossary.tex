% Glossaire

% préfixer le slug de "gls" pour distinguer l'acronyme de la définition.
% \newdefinition{slug}{nom}{source}{texte}

\newdefinition{gls-efqm}{EFQM}{brochure-public-v2020}
{L’EFQM est un organisme innovant, à but non lucratif, qui offre aux organisations et aux professionnels des opportunités de développement, de réseau et d’échanges, des analyses de données et des connaissances inspirantes basées sur une compréhension fine du monde économique actuel.}

\newdefinition{gls-parties-prenantes}{parties prenantes}{parties-prenantes}
{Les parties prenantes sont l'ensemble des personnes physiques et des personnes morales qui sont concernées et qui peuvent influencer les décisions d'une entreprise.[...] Le poids de chaque partie prenante a des répercussions sur la gouvernance d'entreprise. En font notamment partie les employés, les partenaires et les clients.}

\newdefinition{gls-megatrend}{mégatrend}{christian-hohmann}
{Tendance de fond émergente qui influencera plusieurs domaines, tels que la société, la culture, l’économie, la politique, etc. L’impact est tel qu’il définit le monde futur, transforme la société et influence les vies dans les années à venir ainsi que le rythme du changement vers ce futur annoncé.}

\newdefinition{gls-marche}{marché}{larousse}
{Lieu théorique où se rencontrent l'offre et la demande.}

\newdefinition{gls-cloud}{cloud informatique}{ionos-digitalguide}
{Le Cloud (anglais pour « nuage ») fournit de l’espace de stockage, de la puissance de calcul et des logiciels exécutables dans un centre de données distant. Le terme tient compte du fait que le serveur utilisé à cette fin n’est pas directement visible par l’utilisateur, mais caché comme derrière un nuage.}

\newdefinition{gls-chatbot}{chatbot}{definitions-marketing}
{Un chatbot est un robot logiciel pouvant dialoguer avec un individu ou consommateur par le biais d'un service de conversations automatisées pouvant être effectuées par le biais d'arborescences de choix ou par une capacité à traiter le langage naturel.}

\newdefinition{gls-industrie-4-0}{industrie 4.0}{wikipedia-industrie-4-0}
{Le concept d’industrie 4.0 ou industrie du futur correspond à une nouvelle façon d’organiser les moyens de production. Cette nouvelle industrie s'affirme comme la convergence du monde virtuel, de la conception numérique, de la gestion avec les produits et objets du monde réel.}

\newdefinition{gls-jour-du-depassement}{jour du dépassement}{wwf-jdd}
{Date à partir de laquelle l’empreinte écologique dépasse la biocapacité de la planète. Autrement dit, le jour à partir duquel nous avons pêché plus de poissons, abattu plus d’arbres, construit et cultivé sur plus de terres que ce que la nature peut nous procurer au cours d'une année.}

\newdefinition{gls-fiscalite-carbone}{fiscalité carbone}{fisc-carbone}
{La fiscalité carbone est généralement mise en place via une taxe ajoutée au prix de vente de produits ou de services en fonction de la quantité de gaz à effet de serre (GES) qu’ils contiennent (émis lors de leur production et/ou de leur utilisation).}

\newdefinition{gls-radar}{RADAR}{radar}
{La logique RADAR est une méthode d’évaluation dynamique et un outil de management puissant qui fournit une approche structurée pour apprécier la performance d’une organisation.}

\newdefinition{gls-sla}{SLA}{wikipedia-sla}
{Formalisation d’une entente négociée entre client et fournisseur. Il met par écrit l’attente des parties sur le contenu des prestations, leurs modalités d'exécution, les responsabilités des parties, les garanties, c'est-à-dire le niveau de service.}

\newdefinition{gls-r4e}{R4E}{evaluation-efqm-c2e-r4e}
{Évaluation EFQM "Reconnu pour l’excellence", évalue la performance actuelle de l'entreprise.}

\newdefinition{gls-iso-27001}{ISO 27001}{iso-27001}
{Norme ISO sur la sécurité informatique.}

\newdefinition{gls-iso-3166}{ISO 3166}{iso-3166}
{Norme ISO sur les codes de pays.}

\newdefinition{gls-iso-8601}{ISO 8601}{iso-8601}
{Norme ISO sur l'écriture des dates en chaînes de caractères.}

\newdefinition{gls-c0271}{C0271}{c0271}
{Formation Afnor pour devenir évaluateur.}

\newdefinition{gls-cq201}{CQ201}{cq201}
{Formation Afnor sur le nouveau modèle EFQM.}

\newdefinition{gls-c2e}{C2E}{evaluation-efqm-c2e-r4e}
{Évaluation EFQM "Engagé vers l’excellence", évalue les ambitions de l'entreprise.}

\newdefinition{gls-pib}{PIB}{larousse-pib}
{En comptabilité nationale, somme des valeurs ajoutées (biens et services) réalisées annuellement sur le territoire national par les entreprises d'un pays, quelle que soit leur nationalité.}