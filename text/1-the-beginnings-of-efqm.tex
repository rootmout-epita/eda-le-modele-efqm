\section{Les débuts de l'EFQM}

\subsection{Le contexte de sa création}

La fin des années 1970 et la première moitié des années 1980 ont été marquées par une récession économique ayant affecté de nombreux pays développés. Les États-Unis et le Japon sortent rapidement de cette récession grâce à la compétitivité de leurs entreprises. Les pays de l'\acrfull{ocde}, composée alors d’une grande partie de pays d’Europe, sont quant à eux plus touchés, par des taux de chômage élevés et une haute inflation. Cette récession a été causée par le choc pétrolier de 1979, lui-même causé par la révolution iranienne de 1979 et la guerre Iran-Irak qui débuta en 1980\footnote{\mycitation{recession-années-80-wikipédia}}.

En Europe, le chômage reste élevé jusqu’à 1985, avec jusqu’à 2,4 millions de demandeurs d’emplois en France en 1985. L’évolution du \textcolor{custom_blue}{\acrshort{pib}} ne dépasse pas les 2\% pendant plusieurs années (1980~: 1,6\%, 1981~: 1,1\%, 1983~: 1,3\% 1984~: 1,5 \%) et l’inflation dépasse les 13\% par année (1980, 1981).

\definitionbloc{gls-pib}

En 1981, François Mitterrand arrive au pouvoir et revire la politique économique du pays. Malheureusement, cette politique n’est pas plus efficace, et c’est en 1983 que le gouvernement Mauroy instaure le «~tournant de la rigueur~»\footnote{\mycitation{tournant-de-la-rigueur}}, politique économique de relance pour lutter contre le chômage, la fuite des capitaux et l’inflation, porté par Jacques Delors, alors Ministre de l’Économie, des Finances et du Budget.

\subsection{La création de la fondation EFQM}

Un certain nombre de dirigeants veulent alors recréer de l’attractivité auprès de leurs clients, afin de redonner de la compétitivité à leurs entreprises. Ainsi, le 15 Septembre 1988, 14 \acrshort{pdg} de grandes entreprises européennes, sous l’égide de Jacques Delors, devenu président de la Commission européenne, signent une lettre d’intention visant à la création d’une entité contrôlant la qualité et l’excellence des entreprises vis-à-vis de ce qu’elles produisent pour leurs clients. Ces 14 dirigeants sont les suivants~:

\vspace{0.1cm}

\begin{itemize}[itemsep=0.1cm]
    \item Umberto Agnelli - \textit{Fiat Auto SpA}
    \item Carlo De Benedetti - \textit{Ing. C. Ollivetti \& C., SpA}
    \item Carl Horst Hahn - \textit{Volkswagen AG}
    \item A. Scharp - \textit{AB Electrolux}
    \item Jan F.A. de Soet - \textit{Koninklijke Luchvaart Maatschappij N.V. (KLM)}
    \item Cornelis Johannes van der Klugt - \textit{N.V. Philips' Gloeilampenfabrieken}
    \item Serge Dassault - \textit{Avions Marcel Dassault-Breguet Aviation}
    \item Heini Lippuner - \textit{Ciba-Geigy AG}
    \item Raymond H. Lévy - \textit{Régie Nationale des Usines Renault}
    \item Francis Lorentz - \textit{Bull SA}
    \item Konrad Eckert - \textit{Robert Bosch GmbH (participant au nom de Marcus Bierich)}
    \item Iain David Thomas Vallance - \textit{British Telecommunications plc}
    \item Fritz Fahrni - \textit{Gebr. Sulzer AG}
    \item R. Morf - \textit{Nestlé SA (participant au nom de Helmut Oswald Maucher)}
\end{itemize}

\vspace{0.5cm}

\begin{figure}[!h]
    \centering
    \includegraphics[width=12cm]{fondateurs-efqm.jpg}
    \caption{Signature de la lettre d’intention de la création de l’\acrshort{efqm}, 15 sept. 1988}
    \label{fig:fondateurs-efqm}
\end{figure}

Les industries principalement représentées sont celles d’automobile, de l’électroménager, de l’aviation, de la pharmacie et outils médicaux et de l’informatique et des télécommunications, tous étant des secteurs affectés par la récession.
Ainsi, en octobre 1989, l’\acrshort{efqm} est créée. C’est alors une fondation européenne, supportée par l’Union Européenne. Lors de sa création, 67 \acrshort{pdg} y adhèrent au modèle proposé. Une équipe d’experts du domaine académique et industriel est alors composée pour évaluer ces entreprises, ce qui amènera à la première remise du Prix de la Qualité Européenne en 1992.

\subsection{La première version}

\begin{figure}[!b]
    \centering
    \includegraphics[width=14cm]{ancien-modele-efqm.png}
    \caption{Première version du modèle \acrshort{efqm}\cite{ancien-modele-efqm}}
    \label{fig:ancien-modele}
\end{figure}

Le premier modèle \acrshort{efqm} \myfigurereference{fig:ancien-modele} se compose de deux grands aspects~: les facteurs, qui impactent les résultats.

Les facteurs sont les éléments sur lesquels une entreprise peut influer afin d’augmenter sa productivité, améliorer son produit, … Ces éléments sont les suivants.

Le « Leadership » consiste en la façon dont l’entreprise est dirigée. Ainsi, la direction d’une entreprise, en montrant l’exemple, peut amener l’entièreté de l’entreprise vers un changement. Ensuite, le «~Personnel~» englobe l’ensemble des employés, qui sont au cœur de la production des produits et des services. La «~Stratégie~» consiste en l’ensemble des objectifs d’une entreprise et les moyens entrepris pour y parvenir. Les « Partenariats et Ressources » sont constitués des produits et services nécessaires à la production des produits et services de l’entreprise, mais également des partenaires fournissant ces ressources. Enfin, les «~Processus~» sont l’ensemble des moyens mis en œuvre au sein de l’entreprise pour produire et fournir les «~Produits \& Services~».

Les résultats, quant à eux, sont les produits et les services fournis aux clients, et comment ceux-ci sont perçus. De ce fait, on retrouve trois catégories de clients : «~Personnel~» pour les résultats propres aux clients particuliers, «~Société~» pour les résultats propres aux entreprises clientes, et «~Client~» pour les résultats communs aux deux premières catégories. Enfin, les «~Résultats Clés~» consistent, entre autres, en les résultats économiques et sociétaux.

Il est important de noter que ce modèle est cyclique. En effet, les facteurs appliqués par une entreprise sur ses produits et services impactent les résultats. Ces résultats vont ensuite être analysés par l’entreprise, afin de modifier les facteurs et de perfectionner ses produits et services. En suivant ce modèle, l’entreprise met en place un système d’amélioration continue.

\subsection{Critiques envers les anciennes versions}

Lors de la refonte du modèle pour la création de la version 2013, une étude a été organisée auprès des entreprises l’utilisant afin de définir les points du modèle méritant des changements et améliorations. Une grande partie des critiques recueillies par cette étude portaient sur le fait que le modèle n’est pas assez modulable en fonction de l’évolution du monde. En effet, remettre au goût du jour un tel modèle est une opération qui prend du temps, de ce fait le modèle doit pouvoir s’adapter à des situations diverses et variées afin de ne jamais se retrouver dépassé.

De même, il a été décrit comme «~n’[étant] plus sexy~». Ce que les entreprises entendent par là, c’est que le modèle a perdu de son attrait auprès de leurs clients, et qu’il est de moins en moins utilisé pour évaluer des entreprises.

\subsection{Les pistes d’amélioration}

Afin d’améliorer le modèle, un certain nombre de pistes ont été proposées par les entreprises, qui mèneront à la dernière version du modèle, celle de 2020.

Tout d’abord, la partie leadership du modèle a été encouragée à être modifiée afin d’être
moins hiérarchique et plus collaborative pour refléter les changements du monde du travail.

Ensuite, le modèle de travail présenté par le modèle devrait être modifié afin de permettre une évolution permanente de celui-ci. En effet, du fait que les besoins des clients sont amenés à évoluer rapidement, le mode de travail se doit d’être agile afin de mieux y répondre.

Enfin, l’approche d'interaction et d’inclusion du personnel dans la société et la culture de l’organisation doit être retravaillée afin de reconnaître la valeur apportée de chacun et le fait que la population d’une entreprise est de plus en plus diverse. De même, un état d’esprit d’innovation doit être insufflé à l’échelle de l’organisation et dans sa culture.

\subsection{La consistance du modèle}

À travers toutes les versions, l’\acrshort{efqm} garde comme important~:

\begin{itemize}[itemsep=0.1cm]
    \item la primauté du client~;
    \item le besoin d’une vision à long-terme, centrée sur les \gls{gls-parties-prenantes}~;
    \item la compréhension des liens de cause à effet entre ce qui motive une organisation à entreprendre une action, la manière dont elle la mène et ce qu’elle obtient comme résultats.
\end{itemize}
