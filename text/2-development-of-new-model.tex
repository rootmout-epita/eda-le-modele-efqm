\section{L'élaboration du modèle 2020}

Le modèle \acrshort{efqm} justifie son existence de l’aide qu’il apporte aux sociétés qui suivent ses guides. Or comme nous le rappelons en introduction, toute entreprise évolue dans un écosystème en perpétuel mouvement et ne pas adapter le modèle \acrshort{efqm} aux évolutions majeures du monde, serait une aberration. Afin de pallier à cela, le comité exécutif évalue régulièrement la possibilité de lancer une opération de renouvellement, ce qui sera le cas en 2018.

\subsection{Analyse de l'ancien modèle}

Avant de commencer les travaux, il fallait cependant avoir conscience de l’image portée au modèle dans les entreprises et comment celles-ci le mettent en place. Il fallait également effectuer une vaste étude pour connaître les attentes des décideurs sur l’adoption d’un tel modèle et identifier avec eux, quels sont les points divergents avec la réalité.

La collecte de toutes ces informations ne pouvait être négligée tant elle est la base sur laquelle doivent s'appuyer les experts pour établir le nouveau modèle. Disposer d’une base incomplète ou mal travaillée, pourrait compromettre la qualité du modèle et aurait de fortes répercussions.

Différentes méthodes ont été employées\footnote{\mycitation{nouveau-modèle-quels-nouveautés}} pour réaliser cette collecte, la première a été l’envoi d’un questionnaire à environ 1000 entreprises. En parallèle, 60 grands leaders ont été interviewés individuellement. Il leur a notamment été posé des questions comme «~qu’est-ce qui vous empêche de dormir le soir~», qui devait révéler leur principale source d’inquiétude. A noter que dans ce panel de dirigeants, certains n’étaient pas adhérents au modèle \acrshort{efqm}, le but étant de comprendre par la même occasion pourquoi certains s’étaient montrés réticents à son encontre.

Enfin un grand nombre d’experts ont eux aussi été sollicités pour donner leur avis sur le modèle. Issues de cabinets qui évaluent les entreprises conformément aux outils à leur disposition, ils sont les plus à même d’identifier les points qui provoquent une amélioration de ceux qui n’ont aucun effet, voire un effet négatif.

\subsection{Les mégatrends}

L’\acrshort{efqm} a également formé et missionné un petit comité, chargé d’identifier les grandes tendances de demain, les \textcolor{custom_blue}{\glspl{gls-megatrend}}.

\definitionbloc{gls-megatrend}

Plusieurs mois ont été nécessaires aux experts pour cerner les \glspl{gls-megatrend}, et comme vous pourrez le constater, ils ne sont pas limités à un seul domaine en particulier même si certains peuvent être liés. Toutes les entreprises ne sont pas directement concernées, mais doivent avoir conscience de ces points pour évaluer la probabilité d'impact sur leurs affaires.

\subsubsection{Manager la diversité démographique et sociale}

Une augmentation de la diversité parmi les employées est à prévoir avec une mondialisation apportant de nouvelles cultures. Par ailleurs, l’âge de départ à la retraite ne cesse d’être repoussé, la disparité homme femme est plus juste, etc., sont autant de changement qui apporte une disparité de rapport avec le travail.

Chacun n’ayant pas les mêmes attentes ou la même vision des choses, les managers doivent se préparer au défi de faire de cette diversité un atout pour la productivité des équipes. Le manager devra se montrer davantage à l’écoute et compréhensif.

\subsubsection{Essor de la technologie et du digital}

De par son omniprésence évidente, l’informatique touche toutes les organisations, autant sur leur fonctionnement interne que sur leurs échanges avec le client. Pour certaines, il s’agit même de leur cœur de métier.

Il y a d’une part les anciennes entreprises qui ont investi beaucoup de capitaux pour se doter d’un \acrshort{si}, aujourd’hui dépassé, et dont le renouvellement est tout aussi onéreux. Et d’autre part, les nouvelles entreprises, généralement des startups qui, sont parfaitement à l’aise avec ces technologies et en tirent pleinement partie. Un clivage entre ces deux groupes est d’autant plus à craindre que la performance d’un bien informatique est assez vite dépassée.

\begin{figure}
    \centering
    \includegraphics[width=12cm]{cloud-market.jpg}
    \caption{Évolution du \gls{gls-marche} du cloud \cite{cloud-market}}
    \label{fig:cloud-market}
\end{figure}

On observe, par le perfectionnement des télécommunications, un attrait prononcé pour la migration vers le \gls{gls-cloud} \myfigurereference{fig:cloud-market}. Les vitesses de connexion permettent de rendre optionnel la présence de serveurs sur site. De plus, en déléguant la gestion des machines, les entreprises ne payent plus que les ressources qu’elles consomment et s’affranchissent de la même manière de la complexité que représente la gestion d’un parc. Les coûts humains et matériels sont eux aussi réduits ce qui, pour de petites sociétés ou des sociétés de taille intermédiaire, n’est pas négligeable.

D’un point de vue consommateur, la demande s’oriente vers une consommation d’avantage à distance et en ligne. Celui-ci souhaite pouvoir faire un maximum d'actions en ligne pour ne pas avoir à faire de pénibles déplacements. Cet aspect se manifeste avec une forte utilisation des \glspl{gls-chatbot} ou encore la croissance exponentielle des sites d’e-commerce.

\subsubsection{Automatisation et entreprise libérée}

Le modèle classique des grandes entreprises à savoir donner le pouvoir aux dirigeants au sommet de la pyramide hiérarchique est un mode de fonctionnement de plus en plus critiqué. Déconnecté de la réalité, individualiste, etc., il laisse place peu à peu à une redistribution de la parole à l’aval de l’organisation. L’employé a besoin de se sentir utile et considéré.

Les managers voient une opportunité dans l’utilisation du découpage en groupes. Dans les différents services, les équipes sont subdivisées en groupes et jouissent d’un plus grand degré d’autonomie pour prendre des décisions. Cette autonomie ne se concentre pas seulement sur le plan opérationnel mais également sur leur vision d’avenir. Quels seront leurs prochains objectifs, quelle place, quelle importance souhaitent-ils occuper à l’avenir sont autant d'aspects qu’il leur revient de décider.

Un autre besoin se fait sentir, celui d’augmenter les communications entre employés. Ces dernières années, les plateformes de collaboration tel que Slack ou Microsoft Teams, sorte d’évolution à l’indémodable e-mail, ont pris leur envol dans les équipes. En laissant de côté l’aspect formel du mail pour se tourner vers une conversation plus réactive, les employés peuvent accélérer leurs échanges et gagner en productivité. Par l’aspect plus chaleureux, il renforce un sentiment de bien-être chez ses utilisateurs.

\subsubsection{Demande accrue en compétences et effet de l'automatisation}

La production est elle aussi en transition notamment avec ce qui est appelé l’\gls{gls-industrie-4-0}. Il s’agit d’une industrie plus connectée, génératrice d’une grande quantité d’informations, et qui, pour s’avérer pertinentes, ont besoin d’être traitées par des technologies basées sur de l’\acrshort{ia}. Mettre en place de tels processus demande une main d'œuvre très qualifiée qui risque de manquer dans les années à venir et doit être anticiper dès à présent.

Selon une étude\footnote{\mycitation{les-metiers-du-futur-n-existent-pas}} publiée par Dell et l’Institut pour le futur, 85\% des emplois de 2030 n’existent pas encore. Conséquence directe du megatrend n°2 avec une forte transition vers le numérique donc l’\gls{gls-industrie-4-0} du paragraphe précédent n’est qu’un exemple.

\notebloc{L'industrie 4.0 est considérée comme la 4ème révolution industrielle depuis 1780 or la France semble être en retard de 40 milliards d'euros selon E.Macron. Voir le document \mycitation{wavestone-industrie-4-0}}

D’autres domaines vont en effet demander plus de main d'œuvre hautement qualifiée, lorsque l’utilisation des nouvelles technologies (impression 3D, \acrshort{vr}, \acrshort{iot}) deviendra une norme dans tout secteur confondu. Une pénurie de ces métiers n’est pas improbable et doit être anticipée dans l’éventualité où elle serait de taille.

\subsubsection{Économie du partage et de la confiance}

Un megatrend qui touche quant à lui davantage les tendances des consommateurs grand public que les processus. Depuis quelques années, une nouvelle branche de consommation basée sur le partage et le service semble prendre forme. Les utilisateurs souhaitent pouvoir profiter d’un bien mais à moindre frais et se tournent donc vers un nouveau type de plateforme.

\begin{figure}[!b]
    \centering
    \includegraphics[width=10cm]{airbnb.jpeg}
    \caption{Évolution du nombre de nouveaux clients sur airbnb.com \cite{airbnb}}
    \label{fig:airbnb}
\end{figure}

En ce qui concerne l’échange, Airbnb constitue à lui seul un exemple de cette mode \myfigurereference{fig:airbnb}. Les utilisateurs peuvent proposer de louer leur appartement à d’autres, avec un tarif souvent très avantageux. Le site compte d’ailleurs aujourd’hui 150 millions de clients.

La location trouve un coup de vigueur avec notamment l’envol de services comme lime en ville, qui permet tout simplement de louer une trottinette à la minute. Il est probable que ce genre de service trouve également acquéreur pour de futures voitures autonomes\footnote{Google développe sa solution Waymo}.

A travers leurs achats ou locations, les utilisateurs tissent une confiance mutuelle grâce aux avis. La transparence est recherchée afin de trouver les meilleurs produits sans être face à de mauvaises surprises. Cette montée en puissance des commentaires oblige les fabricants et commerces de toute taille à surveiller voire prendre part aux avis laissés. En effet, une entreprise ou un produit doté d’un mauvais score, sans répondre aux critiques, verra ses ventes inéluctablement impactées.

\subsubsection{Raréfaction de la ressource}

Les ressources sur terre ne sont pas inépuisables. Certaines ont une quantité exploitable limitée, comme certaines terres rares, d’autres sont limitées en capacités de production annuelle, comme les céréales, l’oxygène, etc.

Malgré cette contrainte inexorable, l’espère humaine ne cesse d’utiliser des ressources pour satisfaire ses moindres besoins. La courbe de la population mondiale démontre une forte évolution depuis quelques années et cette tendance n’est pas prête de s'essouffler selon les experts.

D’un autre côté, le pourcentage de populations pauvres s’affaiblit et le niveau de vie augmente. Bien que ce dernier point constitue une bonne nouvelle, il a pour conséquence de former une augmentation de la demande sur les produits de seconde nécessité, bien plus gourmandes en ressources à produire. Songez simplement à l’énergie nécessaire et aux matériaux utilisés par la fabrication d’un seul smartphone.

\begin{figure}
    \centering
    \includegraphics[width=12cm]{jour-du-depassement.jpeg}
    \caption{Évolution du jour du dépassement \cite{wikipedia-jdd}}
    \label{fig:jdd}
\end{figure}

Cette consommation frénétique peut se mesurer avec la date du \textcolor{custom_blue}{\gls{gls-jour-du-depassement}} qui ne fait qu’avancer d’années en années vers le mois de janvier \myfigurereference{fig:jdd}. Il était fixé au 29 décembre en 1970, et aujourd’hui, soit 50 plus tard, ce jour est fixé au 29 juillet.

\definitionbloc{gls-jour-du-depassement}

Les entreprises se doivent de trouver des alternatives ou revoir leur modèle de production et d'acheminement des produits, sinon il ne sera pas possible de continuer à cette allure pendant de nombreuses années.

\subsubsection{Incertitude géopolitique et localisation}

Avec une main d'œuvre bien plus abordable et l’apparition du conteneur (maritime), la délocalisation dans les pays de l’Asie était devenue monnaie courante dans les années 1970. Aujourd’hui cependant, ces pays, “usines du monde” entrent dans une récession et les pays développés sont pris d’une vague de nationalisme. Ceux-ci, comme le démontrent le Brexit ou l’administration Trump, tentent de protéger leur économie et leurs industries par l'instauration de taxes importantes sur les produits importés.

Les entreprises qui ont développé leur chaîne de production à une échelle mondiale se doivent de surveiller les actions gouvernementales afin de prévoir un plan d’action adéquat et ne pas mettre à mal ses capacités. Si un pays semble prendre des décisions pour limiter le transit de marchandise qui rentre dans la chaîne de fabrication, il est important pour cette entreprise de considérer les différentes issues possibles.

Apple, depuis peu, cherche à faire revenir sa production aux USA par exemple, une ambition de taxer les produits importés étant en train de se profiler\footnote{\mycitation{moving-iphone-production-to-usa}}.

En ce qui concerne les consommateurs, ceux-ci sont devenus soucieux de l’origine des produits que ce soit pour des raisons environnementales ou pour encourager l’économie locale.  De nouvelles alliances locales apparaissent et des étiquettes et labels vantent les mérites d’un savoir-faire de proximité encourageant le consommateur à délaisser l'iconique “made in china”.

\subsubsection{Confrontation à des réglementations toujours plus exigeantes}

La tendance au sein des décideurs politiques est de recourir à la mise place de normes pour réguler et contrôler les \glspl{gls-marche} mais également prévenir les risques sur les consommateurs. Garantir que les produits vendus sur le territoire ne nuisent pas à la santé, à l'environnement, font partie des premières préoccupations des populations.

Face à ce fait, les entreprises ne devraient pas rester passives aurait tout intérêt à prendre part aux débats qui élaborent ces normes. L’idée est d’établir un dialogue avec les politiques pour effectuer des propositions, apporter un avis éclairé sur la situation pour favoriser la mise en place de réglementations cohérentes.

Travailler avec ces comités permet également d’effectuer une veille proactive sur les nouvelles mesures et s’y conformer le plus rapidement possible.

\subsubsection{Environnement et changement climatique}

Le dérèglement climatique n’est plus à prouver et des mesures doivent être prises pour lutter contre les différents types de pollutions. Les populations et les gouvernements ont compris qu’il en allait de la qualité de vie des générations futures et le sujet écologique est plus que présent dans les débats actuels.

Ce point rejoint le \gls{gls-megatrend} précédent, car pour réguler l'impact sur la planète des activités humaines, de nouvelles réglementations ont vu le jour. Pour ne citer en Europe que la \textcolor{custom_blue}{\gls{gls-fiscalite-carbone}} qui a comme ambition de favoriser les entreprises prenant des mesures et, à l’inverse, pénaliser les mauvais élèves.

\definitionbloc{gls-fiscalite-carbone}

Les consommateurs sont conscients de ces enjeux et sont eux aussi à favoriser les groupes prenant des actions concrètes contre le réchauffement climatique. Des géants technologiques suivent d’ailleurs cette démarche et travaillent à la neutralité carbone.

\subsection{Le processus de rédaction}

Riche de toutes ces informations, l’\acrshort{efqm} a pu lancer les travaux de mise à jour en juin 2018\footnote{\mycitation{quid-megatrends}} sous la responsabilité d’un comité formé à cette occasion, la dénommée «~core team~». Cette équipe est un ensemble de 14 experts provenant d’entreprises ou de cabinets de conseil issus du monde entier. Leur but, simple sur le papier, était donc de créer un nouveau modèle \acrshort{efqm}, mais surtout de le faire avec les premiers concernés à savoir les entreprises. Pour y parvenir, 5 workshops ont été organisés sur une durée de 9 mois. Ainsi les experts n’étaient pas les seuls impliqués et toutes les idées issues des entreprises avaient vocation à être prises en compte.

Après ces quelques mois pour élaborer la structure du modèle, celui-ci a été soumis à l’\acrshort{efqm} pour validation, avant d’être rédigé et traduit pour être présenté officiellement lors de du forum \acrshort{efqm} le 23 \& 24 octobre 2019 à Helsinki (Finlande).

\subsection{Les changements notables}

Cette nouvelle version a apporté trois évolutions notables. La première que nous pouvons citer concerne l’aptitude à la transformation des processus. De par son importance sur la pérennité d’une entreprise, l’étude et l’anticipation de son \gls{gls-marche} a toujours été considéré dans les anciens modèles, l’exemple en introduction le démontre. Aucune méthode n’était cependant donnée pour opérer correctement ces changements pendant que l’exploitation continue. En effet, modifier des processus alors qu’ils sont en cours d’utilisation n’est pas chose évidente et peut rebuter certains à le faire.

Le second grand point est l’agrandissement du spectre des niveaux hiérarchiques représentant le leadership. Un megatrend (le 3) a mis en évidence une subsidiarisation des décisions d’entreprise et de ce fait, les personnes proches de la réalité du terrain reprennent une place dans les prises de décision. Cet élément est tout à l’honneur de l’entreprise car elle engendre des décisions qui sont sous le contrôle de ceux qui devront la mettre en place et qui sauront anticiper d'éventuels problèmes. Cette nouvelle disposition demande la mise en place d’une collaboration transverse, ce qui est pris en compte dans ce nouveau modèle donc.

La nouvelle version modifie également la vision que l’entreprise porte sur elle-même. Anciennement la stratégie était le point de départ des actions à entreprendre, désormais, la raison d’être et la culture de l’entreprise sont reportées sous les projecteurs et devraient constituer la base. La question pour chaque organisme serait de se poser la question sur la raison de son existence, quelle valeur apporte-t-elle à ses \gls{gls-parties-prenantes}. Le terme \gls{gls-parties-prenantes} désigne non seulement les clients et les actionnaires mais aussi les employés. En effet, la version 2020 reconsidère à la hausse la place des employés, l’aspect social et environnemental des actions de l’entreprise.

\notebloc{
    La Fondation encourage les organisations
    à soutenir les objectifs des Nations Unies.
    Ainsi la nouvelle version du Modèle \acrshort{efqm}
    s’appuie sur le pacte mondial des nations unis et les 17 objectifs de développement
    durable des nations unies. Voir page 5 de la \mycitation{brochure-public-v2020}.
}

Les études ont démontré l’importance pour elles de gagner en développement durable. Cela est dans son intérêt, car pour rappel, le développement durable a pour définition d’être une vision stratégique considérant l’aspect social et environnemental d’une importance équivalente à l’économie. Respecter cette pérennité veut donc dire garantir les bénéfices dans le temps.

\clearpage

Comme vous avez pu le voir dans cette partie, le modèle \acrshort{efqm} est ancré dans bon nombre d’entreprises, mais surtout prend en compte ces dernières pour proposer une évolution qui soit la plus en phase avec les attentes du monde tel qu’il évolue.

Maintenant que vous connaissez les principales modifications qui ont été opérées dans la version 2020, il est temps de vous en présenter le contenu, ce que nous allons faire dans la partie qui suit.
