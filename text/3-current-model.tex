\section{Composition du modèle 2020}

\begin{figure}[]
    \centering
    \includegraphics[width=13cm]{efqm-2020.png}
    \caption{Modèle \acrshort{efqm} 2020}
    \label{fig:efqm-2020}
\end{figure}

Contrastant avec son prédécesseur, le modèle \acrshort{efqm} de 2020 se veut plus simple et s'organise autour de trois grands axes. Chacun d'eux sont divisés en deux ou trois critères, qui regroupent d'une part les attentes à respecter pour les organisations souhaitant utiliser le modèle et d'autre part des conseils pour remplir les attentes précédemment énoncés. Notez que ces trois axes ne sont pas indépendant, ils sont accompagnés de flèches comme représenté sur le schéma du modèle, et se complètent afin d'amener l'entreprise vers l'excellence.

\subsection{Axe orientation}

L’axe opérations du modèle \acrshort{efqm} englobe 2 critères.

\subsubsection{Raison d'être, vision et stratégie}

Ce premier critère demande à l'entreprise de reprendre conscience des fondements de son existence. Avant de tracer les grands axes de sa stratégies, il est important de se rappeler le cadre dans lequel nous évoluons et quel est la mission que nous accomplissons. La photographie issue de cette démarche est un premier pas vers la réalisation pratique. En effet, les employés seront plus à même de se sentir investis et travaillant au service d'une cause, et ainsi seront plus productifs pour la création des produits et services proposés auprès des clients. 

Établir et présenter cette vision constitue également un atout au près des personnes souhaitant intégré la structure. Par la visions claire de ce que leur travail pourra apporter ils seront d'autant plus réconforté et rassuré dans l'optique de travailler pour votre société.

Une entreprise qui défini clairement sa raison d'être augmente de plus la porté de sa voix. Il s'agit d'une organisation maîtrisant le plan de route qu'elle partage avec ses \gls{gls-parties-prenantes} et procure de fait un sentiment de confiance chez ses interlocuteurs.

\vspace{1cm}

Attentes~:
\begin{enumerate}
    \item une raison d'être qui inspire~;
    \item une vision ambitieuse~;
    \item une stratégie qui aboutit.
\end{enumerate}

\vspace{1cm}

Mise en œuvre~:

L'entreprise se doit de définir les points suivants~:

\begin{enumerate}
    \item pourquoi l'activité est importante~;
    \item le domaine dans lequel la valeur est créée~;
    \item le cadre de responsabilité de l'activité.
\end{enumerate}

\vspace{1cm}

\subsubsection{Culture et leadership}

Ce second critère s'inscrit dans la lignée du premier. En effet, il reprend l'image que l'entreprise a créée au point précédent, et l'agrémente de toute une culture lui correspondant. Ensuite, afin que cette image soit propagée auprès de tout les employés et que ceux-ci la mettent en œuvre, le leadership, à savoir les dirigeants de l'organisation, se doivent d'incarner des valeurs et des normes qui correspondent à cette image. Le but de cette manœuvre est de promouvoir la réussite et le changement au sein des équipes de travail, mais également à l'échelle de l'entreprise, pour pouvoir plus facilement favoriser la créativité et l'innovation.

\vspace{1cm}

Attentes~:
\begin{enumerate}
    \item valeurs et normes définies~;
    \item définition de l'interaction entre les personnes et les groupes~;
    \item le leadership incarne ces valeurs, normes et interactions.
\end{enumerate}

\vspace{1cm}

Mise en œuvre~:
\begin{enumerate}
    \item développer la culture et les valeurs~;
    \item créer les conditions pour la réussite et le changement~;
    \item favoriser la créativité et l'innovation~;
    \item mobiliser et engager dans la raison d'être, la vision et la stratégie.
\end{enumerate}

\vspace{1cm}

\clearpage

\subsection{Axe opérations}

L’axe opérations du modèle \acrshort{efqm} englobe 3 critères.

\subsubsection{Engagement des parties prenantes}

Le premier constitue l’engagement des \gls{gls-parties-prenantes}. Les \gls{gls-parties-prenantes} représentent tous les groupes de personnes qui sont amenés à interagir avec l’entreprise, que ce soit les clients, des prestataires ou des partenaires. Une bonne entreprise, consciente de l'impact de ces interactions a tout intérêt à engager les \gls{gls-parties-prenantes} dans son activité de valorisation et il est important de les organiser par ordre et par importance.

Généralement il est plus intéressant pour une entreprise de cibler ses clients, cela car ce sont eux qui amènent les contrats, en d’autres termes, les fonds qui font et feront tourner l’entreprise. Il a donc tout intérêt à soigner sa collaboration avec eux. 

Pour s’assurer d’avoir la meilleure collaboration il faut donc être capable d’identifier les besoins des \gls{gls-parties-prenantes} ainsi que leurs attentes. Une fois que cet aspect est correctement cerné, et que l’entreprise à su répondre aux attentes, il faut entretenir la relation de confiance qui s’est instaurée.

Si on est capable de comprendre et remplir précisément le besoin d’un client alors on a déjà créé de la valeur car, en lui apportant pleine satisfaction, ce dernier n'entamera aucune démarches, chronophage, pour changer de fournisseur et nous conservons alors ce partenariat ainsi que ses avantages pécuniers.

Une fois le besoin comblé on vient s'intéresser à la perception que le client en a. En effet, si sur la technique tout s’est parfaitement déroulé, les frictions dans la communication ou la coordination ont peut être créer une expérience déplorable pour la partie prenante qu’il s’efforceront de ne jamais renouveler en allant travailler avec d’autres entreprises. Ces concurrents, qui proposent probablement une solution de moindre qualité mais là où la perception du travail accompli est bien meilleure, se verront comme solution de premier choix . 

Pour améliorer cette perception, il est impératif d’adopter une démarche de collecte active. Le but n’est pas ici d’attendre passivement que les retours nous parviennent en prenant le risque de ne jamais obtenir l'information, mais de collecter de manière proactive les avis et les retours des \gls{gls-parties-prenantes}. 

Ces retours vont ensuite pouvoir être utilisés dans le but de mettre en place un ensemble d’actions appropriées pour sécuriser le futur de l’entreprise. Cela inclut également, par extension, celui de tous ses employés et de toutes les \gls{gls-parties-prenantes}, c'est-à-dire les partenaires de l’entreprise et tous ceux qui en dépendent au sens large.

\vspace{1cm}

Attentes~:
\begin{enumerate}
    \item impliquer les \gls{gls-parties-prenantes} pour les fidéliser.
\end{enumerate}

\vspace{1cm}

Mise en œuvre~:
\begin{enumerate}
    \item identifier les groupes de \gls{gls-parties-prenantes}~;
    \item comprendre leurs besoins et leurs attentes~;
    \item construire, maintenir et développer une relation de confiance~;
    \item collection active de retours~;
    \item mise en place des actions appropriées pour sécuriser son futur.
\end{enumerate}

\vspace{1cm}

\subsubsection{Création de valeur durable}

Le second critère est la création de valeur durable. Cela se fait partiellement par ce qui a été évoqué dans le point précédent ou en introduction à savoir porter une grande attention à l’évolution de son \gls{gls-marche} et évaluer la malléabilité de ses processus pour les conformer aux nouvelles attentes.

Prenons un premier exemple concret, imaginons une entreprise de voirie privée qui a besoin de creuser des tunnels. Dans ce cas, il y a un fournisseur qui dispose des tunneliers nécessaires et qui est le partenaire privilégié de cette entreprise pour déboucher ce genre d’impasse.

Le besoin principal, celui de creuser un tunnel, est toujours  le même, mais il est possible que les standards évoluent au cours du temps.

L’entreprise de voirie peut, à l’avenir, exiger par exemple qu’il soit plus large, plus long que les précédents, qu’il creuse dans des matériaux plus durs, qu’il abrite plus de matériel, plus de câbles, avec une infrastructure plus récente, qu’il puisse être équipé d’antennes pour capter la 4G dans le tunnel pour répondre aux exigences grandissantes des passagers et des normes de sécurité et d’accessibilité.

Une entreprise incapable de mettre à jours son catalogue pour suivre les besoins du \gls{gls-marche} se retrouvera écartée même si c’est la plus compétente dans son domaine et qu’elle entretient des relations et un savoir faire privilégié avec ses partis prenantes. 

Il est donc indispensable de mettre à jour les services qu’une entreprise offre et de toujours proposer les dernières fonctionnalités. Si nous prenons un second exemple avec un fabricant de switch ethernet, si une nouvelle fonctionnalité est absolument requise et qu’elle n’est proposée que par un unique fabricant alors tous les clients vont naturellement vous rediriger vers lui. 

Et donc, pour se faire, de manière relativement similaire au point précédent, une entreprise se doit de se tenir au courant de ce qu’il se passe en collectant les avis et opinions des \gls{gls-parties-prenantes} et être capable d’améliorer ses produits, services et solutions en conséquence. 

\vspace{1cm}

Attentes~:
\begin{enumerate}
    \item la création de valeur durable est vitale pour le succès à long terme de l’organisation ainsi que sa solidité financière.
\end{enumerate}

\vspace{1cm}

Mise en œuvre~:
\begin{enumerate}
    \item cibler les clients~;
    \item s’adapter aux changements des besoins des \gls{gls-parties-prenantes}~;
    \item collecte de leurs avis et opinions~;
    \item amélioration des produits, service et solutions.
\end{enumerate}

\vspace{1cm}

\subsubsection{Pilotage de la performance et la conduite de la transformation}

Le troisième axe est le pilotage de la performance et la conduite de la transformation. Il faut à la fois piloter ses activités pour mener ses projets à bout tout en s’adaptant aux changements de son environnement.

Illustrons ce point par un nouvel exemple. Une entreprise d'extraction de pétrole base son économie sur l’extraction d’une ressource non renouvelable et qui, intrinsèquement, se raréfie par la même occasion. 

Il lui faut également être capable de diversifier son secteur d’activité et le transformer pour s’assurer de la pérennité de ses opérations.

Ces changements peuvent être externes, comme des nouvelles lois environnementales ou internes comme la découverte de nouvelles méthodes d’extractions de pétrole plus efficaces. Dans notre cas d’entreprise de tunnel des normes de sécurité toujours plus exigeantes ou des tunneliers de nouvelle génération qui apportent leur lots d’améliorations peuvent être la source de ce changement.

C’est une transformation qui peut également se faire autour de la donnée et les nouvelles technologies. En effet, la donnée est aujourd’hui quelque chose dont l’importance n’est plus à démontrer et qui peut être valorisée au point d’être la principale entrée de fonds de certains services comme Google et YouTube ou les principaux fournisseurs de boite mail en raison du ciblage publicitaire qu’elle permet\footnote{\mycitation{data-market}}.

Beaucoup d’entreprises récoltent de la donnée mais ne l'exploitent pas. Pourtant, pour ceux capables de l’analyser, elle donne des informations très importantes concernant les tendances des utilisateurs et permet de détecter en amont des changements de la manière de consommer. Cela s’applique particulièrement aux entreprises qui s’adressent à un grand nombre de clients ou qui récoltent des  données par le biais de produits vendus.

Il est par exemple possible une route collectant des statistiques précises sur le nombre et le positionnement des passagers dans les voies.

Cela permet au constructeur d’avoir des retours sur l’utilisation des routes et ainsi anticiper les réparations à faire, détecter les points de contention ou les accidents.

Les endroits dans lesquels les utilisateurs viennent se positionner en préférence pour être capable d’anticiper le besoin avant même qu’il ne soit formulé par la société de transport. 

\vspace{1cm}

Attentes~:
\begin{enumerate}
    \item répondre aux attentes de ses clients~;
    \item s’adapter aux changements de son environnement.
\end{enumerate}

\clearpage

Mise en œuvre~:
\begin{enumerate}
    \item sssurer le quotidien~;
    \item préparant le futur~;
    \item l'innovation et la technologie~;
    \item les données~;
    \item utilisation intelligente des ressources.
\end{enumerate}

\vspace{1cm}

\clearpage

\subsection{Axe Résultats}

Le modèle \acrshort{efqm} est utilisé dans des entreprises qui ont déjà su prendre leur envol et qui disposent donc d’une force de vente et de consommateurs. Cette section Résultats va s’intéresser aux conséquences des stratégies passées et quels auront été leurs impacts. Deux points sont à prendre en compte, le premier concerne la perception des partis prenants, le second les performances et les liens de cause à effet qui les entraînent. 
 
\subsubsection{Perceptions des parties prenantes}
 
Ce premier point encourage les entreprises à s’intéresser à l’image que leur stratégie leur confère auprès de leurs partenaires. Il s’agit d’un point sur lequel des objectifs sont fixés et il est important de faire une inspection pour deux raisons. Premièrement, connaître sa position revient à savoir le degré de liberté permissible dans les prochaines. Dans un second temps, constater une différence entre les objectifs et la réalité est un témoin évocateur d’un problème de mise en pratique à corriger. Les études sont-elles fausses ou les stratégies ne sont-elles pas appliquées correctement ?
 
Prenons au hasard l’exemple d’un opérateur de télécommunication qui tente de se placer dans un \gls{gls-marche} premium. Celui-ci peut se doter d’une assistance rapide, d’une infrastructure avec une bonne \textcolor{custom_blue}{\acrshort{sla}}, ou encore de campagnes marketing, le tout pour vanter sa haute qualité et proposer en contrepartie des prix plus forts.

\definitionbloc{gls-sla}

Si ces clients ne le perçoivent pas comme tel, quel que soit la raison, ses offres ne sembleraient pas pertinentes et un concurrent sur le même \gls{gls-marche} serait en passe de prendre la main.

\clearpage

Il est important de rappeler que les clients ne sont pas les seuls partis prenants et que tous doivent être intégrés à cette évaluation. Un fournisseur qui a un attrait plus important pour un concurrent pourrait dégrader la chaîne de production, ou des investisseurs insatisfaits des résultats risquent de revoir à la baisse leur soutien pécuniaire.

\vspace{1cm}

Attentes~:
\begin{enumerate}
    \item capable de reconnaître la valeur ajoutée qui saura intéresser les \gls{gls-parties-prenantes} clés et ainsi orienter la stratégie en conséquence~;
    \item utilise les données sur les performances actuelles afin de réaliser une projection prévisionnelle sur le futur~;
    \item garde une veille sur la perception des partis prenants et détecter ainsi toute fluctuation annonçant un changement du \gls{gls-marche}.
\end{enumerate}

\vspace{1cm}

Mise en œuvre~:
\begin{enumerate}
    \item résultats de perceptions clients~;
    \item résultats de perceptions du personnel~;
    \item résultats de perceptions \gls{gls-parties-prenantes} économiques et institutionnelles~;
    \item résultats de perceptions société~;
    \item résultats de perceptions partenaires et fournisseurs.
\end{enumerate}

\vspace{1cm}

\subsubsection{Performances stratégiques et opérationnelles}
 
Inutile de préciser à une entreprise d’utiliser ses données financières pour juger de sa santé. Dans les grands groupes, il est cependant à rappeler qu’un tri s’impose pour choisir les informations dont le caractère est suffisamment pertinent pour une utilisation dans les analyses. Demander aux experts intervenants dans la société de traiter les informations sur un large spectre pourrait retarder le délai avant qu’une information importante soit reportée aux dirigeants.
 
Les leaders doivent également se rappeler que mesurer les objectifs atteints a une propension à donner l’état du dynamisme de ses équipes. Ne pas remplir ses objectifs à temps peut être sous-jacent d’un manque de formation ou d’effectif des équipes et doit être pallier au plus vite.

\vspace{1cm}

Attentes~:
\begin{enumerate}
    \item utiliser des indicateurs financiers~;
    \item comprendre le lien entre perception des \gls{gls-parties-prenantes} et performances. Prévoir l’évolution de ses performances~;
    \item choisir les meilleurs indicateurs issus de ses \gls{gls-parties-prenantes} pour surveiller l’évolution de ses objectifs stratégiques~;
    \item identifier l’origine des changements de la performance de l’entreprise. En effectuer la veille et ajuster sa stratégie en fonction~;
    \item anticiper précisément ses performances futures grâce au listage des objectifs atteints.
\end{enumerate}
 
\vspace{1cm}

Mise en œuvre~:
\begin{enumerate}
    \item les réalisations en termes d’atteinte de sa raison d’être et de création de valeur durable~;
    \item la performance financière~;
    \item le respect des attentes de ses \gls{gls-parties-prenantes} clés~;
    \item la réalisation des objectifs stratégiques~;
    \item les réalisations dans le pilotage de la performance~;
    \item les réalisations dans la conduite de la transformation~;
    \item les mesures prédictives pour le futur.
\end{enumerate}
