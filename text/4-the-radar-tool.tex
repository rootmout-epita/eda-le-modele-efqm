\section{L'outil de diagnostique RADAR}

Un des outils mis à disposition par EFQM pour mener et contrôler des stratégies d’entreprise (qui peuvent être autre que le modèle EFQM) est l’outil  \acrshort{radar} \myfigurereference{fig:radar}. C’est une méthode d’évaluation dynamique de la performance d'une entreprise ou une partie de cette entreprise. On qualifie cette méthode de dynamique car elle permet l’évaluation de l’entreprise en continu.

\definitionbloc{gls-radar}

\begin{figure}
    \centering
    \includegraphics[width=12cm]{radar.png}
    \caption{Structure de l'outil RADAR \cite{radar}}
    \label{fig:radar}
\end{figure}

\subsection{Les différentes étapes de RADAR}

Dans un premier temps, il est impératif de définir la cible. En effet c’est un modèle dirigé par les Résultats attendus comme toute entreprise digne de ce nom doit l’être.
Le premier R de RADAR est dont attribué au mot résultats

Dans un second temps, les Approches pour y parvenir sont développées.
Cela correspond au A de RADAR. 
Il s’agit, compte tenu de la situation de départ, de choisir les meilleures méthodes pour appliquer des réformes sans engager une apparition de dégâts collatéraux. Ces approches se doivent d’être les plus pertinentes et les plus efficaces pour parvenir aux objectifs  au point précédent. 

Dans un troisième temps, ces approches sont Déployées.
Le D de RADAR est réservé pour ce mot.
En effet, dans une entreprise, il est facile de mettre en place une structure très organisée sur un tableau blanc mais en pratique ces règles sont difficiles à appliquer car une entreprise est constitué d’employés qui ne sont pas des machines, qui ont tous des manières différents de fonctionner.

Les processus de fonctionnement internes ne sont pas fixes et rarement clairement définis. Il y a de plus le facteur humain qui rentre en jeu, toutes les règles ne seront pas appliquées à la lettre et il faut réagir dans ce cas. Notez que le problème peut aussi bien venir des employés que des processus eux-mêmes. Si une personne sur le terrain a dévié de la route tracée, une raison légitime peut en expliquer la cause. 

Pour s’assurer de l’efficacité du modèle, il est important de s’assurer du respect strict des approches qui ont été mises en place. Cela ne veut pas dire que ces approches sont forcément les meilleures mais dans le cas où des difficultés sont rencontrées il est important de les corriger et non laisser à chacun la responsabilité de les faire fonctionner à leur manière.

Cela nous amène donc à notre dernier point, Evaluer et Ameliorer ces approches. L'Apprentissage est le second A de RADAR. C’est grâce au retours que l’on a  pu obtenir grâce au déploiement systématique et leur mise en œuvre pleine et entière que l’on peut ensuite les améliorer. 

En effet, si certaines entreprises sont parvenu à obtenir une position dominante sur  le marché grâce à leur approches, aucune ne peut prétendre disposer de la méthode parfaite ce qui implique une possible amélioration.

Revoir est le dernier R de RADAR, on recommence au début de manière perpétuelle en utilisant les apprentissages acquis pour améliorer itérativement les processus mis en place pour augmenter la performance de l’organisation. 

\subsection{Les fins de son utilisation}

C’est donc une méthode qui peut être mise en place en interne dans le but d’augmenter la performance en analysant en continu les résultats mais qui va également être utilisée pour noter l'entreprise. En effet, des évaluateurs externes peuvent effectuer une notation de l’entreprise en utilisant les grilles de notation RADAR qui vont venir s'intéresser à la conformité et l’étendue de la mise en place du modèle RADAR dans divers aspects de l’organisation. 

Cette notation, si le score obtenu est important, permet d’obtenir de la reconnaissance. En effet, un score suffisant permet d’obtenir une certification délivrée par l’Afnor, organisme reconnu.