\section{Les prix et récompenses}

\subsection{L'Afnor}
\begin{figure}
    \centering
    \includegraphics[width=7cm]{afnor.jpeg}
    \caption{Logo de l'Afnor}
    \label{fig:afnor}
\end{figure}

L’\acrshort{afnor} est l’\acrlong{afnor}. Il s’agit concrètement d’une instance locale qui représente la France dans des comités d'organisation européens ou mondiaux comme le \acrfull{cen}.  Elle représente également la France auprès de l’\acrfull{iso}.

Il s’agit d’une association dont les actions sont visibles pour nous, en tant que consommateurs, au quotidien. C’est elle qui est chargée du contrôle des normes européennes ou françaises sur les produits qui font notre foyer. Deux d’entre eux sont par ailleurs très répandus, le logo de \acrfull{ce} et le logo de conformité aux \acrfull{nf} sont obligatoires pour les produits souhaitant être disposés à la vente sur l'hexagone. Un objet fabriqué estampillé de ce label est alors de qualité suffisante pour ne pas mettre en danger les consommateurs.

\begin{figure}[!h]
    \centering
    \subfloat[\centering Label NF]{{\includegraphics[width=3cm]{nf.png} }}%
    \qquad
    \subfloat[\centering Label CE]{{\includegraphics[width=3cm]{ce.jpeg} }}%
    \caption{Logos des sous-marques de l'Afnor}%
    \label{fig:nf-ce-logo}%
\end{figure}

\clearpage

Cette association a été créée en 1926 et est une association d'intérêt général et est reconnue d'intérêt public\footnote{\mycitation{afnor-association}}.

Cela veut dire que ce n’est pas une entreprise privée qui a été créée dans le but de vendre de la formation une fortune aux entreprises avec pour objectif de générer un maximum d’argent pour ses actionnaires.

Il nous faut également considérer que l’argent récolté par les formations va permettre de servir à l’élaboration et la mise en place de nouvelles normes qui bénéficient à l’ensemble de la population pour une meilleure sécurité au quotidien, aussi bien au niveau des protection en voiture, de la qualité des matériaux de construction, la sécurité des appareils électriques que la réduction de la consommation et la pollution individuelle en facilitant la conceptions de produits compatibles. 

Ces normes touchent aussi les systèmes informatiques. Par exemple, quelques-unes au niveau mondial sont bien connues, comme la norme \gls{gls-iso-27001}\footnote{\mycitation{iso-27001}} pour la sécurité informatique, la norme \gls{gls-iso-3166}\footnote{\mycitation{iso-3166}} pour les noms de pays, par exemple dans les noms de domaine ou encore la norme \gls{gls-iso-8601}\footnote{\mycitation{iso-8601}} pour le format d’écriture des dates en chaînes de caractères.

\subsection{Formation des évaluateurs}

Noter une entreprise sur sa conformation avec le modèle EFQM est effectuée par des évaluateurs et des formations pour le devenir sont disponibles. Par exemple, la formation \gls{gls-c0271}\footnote{\mycitation{iso-8601}} pour devenir évaluateur proposée sur Paris est facturée à 2400€ HT. Cela se présente sous la forme d’un atelier de 21h sur trois jours et qui vous permet d’obtenir une certification.

Nous nous permettons de noter que pour ce prix les repas sont gracieusement offerts. 

Des formations qui se limitent au fonction du dernier modèle EFQM sont également proposées. Et il est dès lors envisageable d’avoir recours à la seconde offre proposée.

La formation \gls{gls-cq201}\footnote{\mycitation{cq201}}, qui est également requise pour devenir évaluateur, est proposée pour la modique somme de 1250€ HT. Elle ne dure que 14h sur deux jours mais représente tout de même l’occasion d’avoir deux  repas offerts. 
 
Une fois les formations validées, il est ensuite possible d’évaluer des entreprises, ou alors faire évaluer l’entreprise qui vous emploi. 

\subsection{Les évaluations}

Pour ce faire, deux évaluations sont proposées. Dans un premier temps l’évaluation \gls{gls-c2e}\footnotemark: 
Elle permet de certifier que même si l’entreprise cible n’a pas complètement appliqué le modèle, elle est dans la  bonne direction sur le chemin pour le faire.
 

\begin{itemize}[label=\textbullet, itemsep=0.4cm]

    \item Dans un premier temps une auto évaluation est réalisée dans le but de comprendre vos performances actuelles ;
    
    \item Dans un second temps la méthode RADAR est utilisée pour déterminer les actions d'amélioration à mettre en place en priorité, vous devez en obtenir au moins trois ;
    
    \item Ensuite un évaluateur EFQM se déplace sur le site de l’entreprise pour évaluer la  pertinence des actions qui ont été entreprises ainsi que la capacité de l’entreprise à les mettre en place ;

    \item L’entreprise obtient ensuite un retour de l’évaluation avec des pistes d’améliorations et des commentaires. Ce n’est donc pas uniquement une évaluation qui a pour but de vous mettre sur un barème mais vraiment également de fournir des retours détaillés qui vont pouvoir être utilisés pour permettre à l’entreprise d'améliorer les choses ;

    \item Et finalement l’entreprise obtient un rapport complet ainsi que le diplôme de reconnaissance EFQM délivré par l’Afnor ;

\end{itemize}

Une fois que l’entreprise a complètement appliqué le modèle EFQM à ses activités, elle peut prétendre à l’évaluation \gls{gls-r4e}\footnotemark[\value{footnote}] “Reconnu pour l’Excellence”.
\footnotetext{\label{efqm-evaluation}\mycitation{evaluation-efqm-c2e-r4e}}

\begin{itemize}[label=\textbullet, itemsep=0.4cm]

    \item Dans un premier temps, un document décrivant le fonctionnement et l’organisation de l’entreprise au niveau managérial est rédigé. Il sera lu par les évaluateurs qui seront sélectionnés en fonction de la taille de l’entreprise et des enjeux ;
    
    \item Un responsable d’évaluation est désigné, il est responsable du bon déroulement de toute l’opération et offre un suivi personnalisé à l’entreprise ;
    
    \item Dans un troisième temps, une équipe d'évaluation intervient en interviewant l’équipe pour analyser le fonctionnement, les approches, le déploiement et les résultats obtenus par l’entreprise ;
    
    \item Une réaction à chaud de l'équipe ainsi qu’un document résumant les scores de l’évaluation établis par consensus de l’équipe d’évaluation est fournie à l’entreprise ;
    
    \item Finalement le rapport final avec le diplôme de reconnaissance et le logo avec la notation finale entre trois et cinq étoiles est remise à l’entreprise ;

\end{itemize}
 
Dans les deux cas, si l'évaluation s'est correctement déroulée, l'entreprise obtient la reconnaissance de l’avoir passé. C’est un titre qui est partagé avec de grosses entreprises comme Renault, Nestlé, Dassault, Phillips

\notebloc{Les entreprises primées sont listées sur le site Global Excellence Index \cite{global-excellence-index}}
