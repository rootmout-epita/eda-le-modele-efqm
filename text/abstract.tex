% Abstract

\thispagestyle{noheader}

\begin{abstract}

\noindent L’EFQM (European Foundation for Quality Management) est une fondation créée en 1989 par 67 grands dirigeants épris de l’ambition d’aider les entreprises européennes, impactées alors par une grande récession économique. Cet organisme est responsable du modèle EFQM, sorte de guide des bonnes pratiques à adopter pour tendre vers la qualité et l’excellence.
La notion de qualité et d’excellence est attribuée aux entreprises conduisant une démarche d’amélioration dans tous les domaines. Celle-ci répond d’une grande performance et parvient à garder ce niveau dans le temps malgré les fluctuations externes (sociales, politiques, économiques, etc).
Ce modèle a été réédité à de multiples reprises, travail réalisé consciencieusement sous la tutelle d’experts et l’aide de grandes entreprises, lui conférant ainsi une pertinence d’actualité malgré ses 30 ans d'âge.
Être estampillé EFQM représente un atout pour se démarquer de la concurrence, un tel label signifiant à vos parties prenantes que vous êtes soucieux d’une collaboration de qualité et axé sur le développement durable. De grandes sociétés ont d’ailleurs fait ce choix, à l’instar de Renault, Nestlé, Dassault ou encore Phillips.

\vspace{1cm}

\color{custom_blue}
\noindent\makebox[\linewidth]{\rule{1cm}{2pt}}
\color{black}

\vspace{1cm}

\noindent [RISSON PART]

\end{abstract}