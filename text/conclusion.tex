\section*{Conclusion}   % no number on table of content
\addcontentsline{toc}{section}{Conclusion}

Comme nous l’avons vu ensemble, le modèle \acrshort{efqm} s’est formé à la suite d’un manque de compétitivité de l’Europe face aux autres puissances mondiales. Fort de leur expérience dans le domaine, les grands dirigeants fondateurs du modèle ont su construire un guide des bonnes pratiques qui s’est révélé efficace et a remédié au manque de compétitivité de certaines structures.
 
Aujourd’hui, après 30 ans d'existence, il reste largement utilisé grâce à l’investissement conséquent de l’\acrshort{efqm} sur un travail de veille et de collecte des résultats auprès de ses tenants. La valeur que représente le modèle est suffisamment importante pour que de grands groupes, de niveau international, se servent de sa vision afin d’édifier leurs stratégies, pour ne citer que Bosch ou BMW en exemple. Tous ces grands groupes sont à eux seuls le témoignage de l'intérêt porté au modèle et illustrent à quel point il aspire à un avenir certain.
 
En 2021, il est difficile de ne pas évoquer la pandémie mondiale tant elle est importante, au point de justifier le pléonasme employé au cours de cette phrase. Pourtant, vous pouvez le constater, aucune allusion n’a été faite en amont de ce document pour la simple raison que nous souhaitions vous retranscrire les éléments du modèle sans apporter de spéculations. L’ouverture d’une conclusion a en revanche une liberté qui se prête à ce que nous donnions nos pistes de réflexion personnelle sur le futur du modèle.
 
L’\acrshort{efqm} ayant à cœur de respecter les grandes tendances, il est fort probable qu’une réédition soit demandé incessamment sous peu. En effet, bien que la crise engendrée par le Sars-CoV-2 soit passagère, elle impactera de nombreux domaines de manière durable en modifiant inéluctablement les \glspl{gls-megatrend}.
 
Prenons le secteur tertiaire, pour exemple, dont les méthodes de travail ont été largement bousculées et repensées pour s’adapter aux confinements. Des outils, jusque-là peu connus, ont gagné en importance et se sont démocratisés comme cela a pu être le cas pour Slack, Zoom ou Trello. Ce sont autant de solutions qui ont séduit les entreprises et leur ont démontré qu’une présence constamment sur place de leurs employés n’était pas obligatoire et pouvait même être bénéfique à leur bien-être.
 
Cette opportunité à travailler de chez soi, a également un impact social, elle remet en question la manière dont les grandes agglomérations ont été bâties. Le lieu de travail pouvant être ignoré lors de la recherche d’un logement, les couronnes périurbaines, cités dortoirs des travailleurs pourraient s’avancer sur un déclin au bénéfice des ruralités. 

Pourquoi s’installer dans un petit logement en île de France lorsque pour le même tarif vous pourriez avoir une maison à la campagne~?
 
Les régions de France sont d’ailleurs confiantes en ce qui concerne l’arrivée d’un tel phénomène. Depuis quelques mois, de nombreuses publicités pullulent dans les transports en communs, toutes ayant comme points commun de vanter le bien être conféré par la vie au vert. Travailler en campagne est d’autant plus possible grâce au développement du TGV et de la fibre sur tout le territoire\footnote{\mycitation{plan-france-tres-haut-debit}}.
 
D’autres actualités, à moindre mesure, ont impacté les entreprises comme le fut le blocage du canal de Suez par l’Ever Given\footnote{\mycitation{le-blocage-du-canal-de-suez}}. L’atteinte de ce point critique dans l’acheminement des marchandises a rappelé à quel point les chaînes de productions issues de la mondialisation étaient fragiles. Elles sont d’un bénéfice sur le plan financier, mais il est utile de prévoir des solutions de secours dans l’éventualité où une étape de la chaîne se retrouve en dysfonctionnement.
 
Vous en conviendrez, le monde ne cesse d’évoluer au gré des crises et des innovations, charriant avec lui les entreprises dont le commerce et l’économie reposent intégralement sur ses tendances. Aucune d’entre elles ne peut se targuer de posséder une stratégie d’avenir parfaite, mais toutes peuvent recourir aux bonnes pratiques du modèle \acrshort{efqm} pour tendre opiniâtrement vers de meilleures performances.
 
Nous espérons que la lecture de ce document aura répondu à vos attentes et que vous l’aurez perçu comme étant de qualité et d’excellence.

\clearpage

\section*{Conclusion (EN)}   % no number on table of content
\addcontentsline{toc}{section}{Conclusion (EN)}

As we have seen together, the \acrshort{efqm} model was formed as a result of Europe's lack of competitiveness compared to other world powers. With their experience in the field, the great leaders who founded the model were able to build a guide to good practices that proved effective and remedied the lack of competitiveness of certain structures.
 
Today, after 30 years of existence, it is still widely used thanks to the significant investment of \acrshort{efqm} in monitoring and collecting results from its supporters. The value that the model represents is sufficiently important that large international groups use its vision to build their strategies, for example Bosch or BMW. All these large groups are in themselves a testimony of the interest in the model and illustrate to what extent it aspires to a certain future.
 
In 2021, it is difficult not to mention the global pandemic, as it is so important that it justifies the pleonasm used in this sentence. However, as you can see, no allusion has been made upstream of this document for the simple reason that we wanted to transcribe the elements of the model without speculation. The opening of a conclusion, on the other hand, is a freedom that lends itself to us giving our personal thoughts on the future of the model.
 
As \acrshort{efqm} is committed to respecting the major trends, it is highly likely that a new edition will be requested shortly. Indeed, although the crisis generated by the Sars-CoV-2 is temporary, it will have a lasting impact on many areas by inevitably modifying the megatrends.
 
Let's take the service sector, for example, whose work methods have been largely shaken up and rethought to adapt to lockdowns. Tools that were previously little known have gained in importance and have been democratized, as was the case for Slack, Zoom or Trello. These are all solutions that have seduced companies and shown them that a constant presence of their employees was not mandatory and could even be beneficial to their well-being.
 
This opportunity to work from home also has a social impact, calling into question the way in which large cities have been built. As the place of work can be ignored when looking for a home, the suburbs, dormitory cities for workers, could go into decline to the benefit of rural areas. Why settle down in a small appartment in Manhattan when for the same price you could have a large house in the countryside ?
 
The regions of France are confident that such a phenomenon will occur. For a few months now, numerous advertisements have been popping up in public transport, all of them having in common the idea of the well-being of living in the countryside. Working in the countryside is even more possible thanks to the development of the TGV and the fiber throughout the territory\footnote{\mycitation{plan-france-tres-haut-debit}}.
 
Other news, to a lesser extent, have impacted companies such as the blocking of the Suez Canal by Ever Given\footnote{\mycitation{le-blocage-du-canal-de-suez}}. The reaching of this critical point in the routing of goods has reminded us how fragile the production chains resulting from globalization are. They are financially beneficial, but it is useful to have backup solutions in the event that one stage of the chain malfunctions.
 
You will agree that the world is constantly evolving through crises and innovations, bringing with it companies whose businesses and economies are based entirely on these trends. None of them can claim to have a perfect strategy for the future, but all of them can use the best practices of the \acrshort{efqm} model to stubbornly strive for better performance.
 
We hope that the reading of this document will have met your expectations and that you will have perceived it as being of quality and excellence.
