\section*{Conclusion}   % no number on table of content
\addcontentsline{toc}{section}{Conclusion}

Comme nous l’avons vu ensemble, le modèle EFQM s’est formé à la suite d’un manque de compétitivité de l’Europe face aux autres puissances mondiales. Fort de leur expérience dans le domaine, les grands dirigeants fondateurs du modèle, ont su construire un guide des bonnes pratiques qui s’est révélé efficace et a remédié au manque de compétitivité de certaines structures.
 
Aujourd’hui, après 30 ans d'existence, il reste largement utilisé grâce à l’investissement conséquent de l’EFQM sur un travail de veille et de collecte des résultats auprès de ses tenants. La valeur que représente le modèle est suffisamment importante pour que de grands groupes, de niveau international, se servent de sa vision afin d’édifier leurs stratégies pour ne citer que Bosch ou BMW en exemple. Tous ces grands groupes sont à eux seuls le témoignage de l'intérêt porté au modèle et illustrent à quel point il aspire à un avenir certain.
 
En 2021, il est difficile de ne pas évoquer la pandémie mondiale tant elle est importante, au point de justifier le pléonasme employé au cours de cette phrase. Pourtant, vous pouvez le constater, aucune allusion n’a été faite en amont de ce document pour la simple raison que nous souhaitions vous retranscrire les éléments du modèle sans apporter de spéculations. L’ouverture d’une conclusion en revanche à une liberté qui se prête à ce que nous donnions nos pistes de réflexion personnelle sur le futur du modèle.
 
L’EFQM ayant à cœur de respecter les grandes tendances, il est fort probable qu’une réédition soit demandé incessamment sous peu. En effet, bien que la crise engendrée par le Sars-CoV-2 soit passagère, elle impactera de nombreux domaines de manière durable en modifiant inéluctablement les megatrends.
 
Prenons le secteur tertiaire, pour exemple, dont les méthodes de travail ont été largement bousculées et repensées pour s’adapter aux confinements. Des outils, jusque-là peu connus, ont gagné en importance et se sont démocratisés comme cela a pu être le cas pour Slack, Zoom ou Trello. Ce sont autant de solutions qui ont séduit les entreprises et leur ont démontré qu’une présence constamment sur place de leurs employés n’était pas obligatoire et pouvait même être bénéfique à leur bien-être.
 
Cette opportunité à travailler de chez soi, a également un impact social, elle remet en question la manière dont les grandes agglomérations ont été bâties. Le lieu de travail pouvant être ignoré lors de la recherche d’un logement, les couronnes périurbaines, cités dortoirs des travailleurs pourraient s’avancer sur un déclin au bénéfice des ruralités. Pourquoi s’installer dans un petit logement en ile de France lorsque pour le même tarif vous pourriez avoir une maison dans ?
 
Les régions de France sont d’ailleurs confiantes en ce qui concerne l’arrivée d’un tel phénomène. Depuis quelques mois, de nombreuses publicités pullulent dans les transports en communs, toutes ayant comme points commun de vanter le bien être conféré par la vie en campagne. Travailler en campagne est d’autant plus possible grâce au développement du TGV et de la fibre sur tout le territoire\footnote{\mycitation{plan-france-tres-haut-debit}}.
 
D’autres actualités, à moindre mesure, ont impacté les entreprises comme le fut le blocage du canal de Suez par l’Ever Given\footnote{\mycitation{le-blocage-du-canal-de-suez}}. L’atteinte de ce point critique dans l’acheminement des marchandises a rappelé à quel point les chaînes de productions issues de la mondialisation étaient fragiles. Elles sont d’un bénéfice sur le plan financier, mais il est utile de prévoir des solutions de secours dans l’éventualité où une étape de la chaîne soit en dysfonctionnement.
 
Vous en conviendrez, le monde ne cesse d’évoluer au gré des crises et des innovations, charriant avec lui les entreprises dont le commerce et l’économie reposent intégralement sur ses tendances. Aucune d’entre elles ne peut se targuer de posséder une stratégie d’avenir parfaite, mais toutes peuvent recourir aux bonnes pratiques du modèle EFQM pour tendre opiniâtrement vers de meilleures performances.
 
Nous espérons que la lecture de ce document aura répondu à vos attentes et que vous l’aurez perçu comme étant de qualité et d’excellence.

\color{custom_blue}
\noindent\makebox[\linewidth]{\rule{1cm}{2pt}}
\color{black}

\vspace{1cm}

\noindent [RISSON PART]