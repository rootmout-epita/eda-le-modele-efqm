\addcontentsline{toc}{section}{Introduction}

\section*{Introduction}

Le monde évolué que nous connaissons a fait émerger des entreprises géantes qui brassent, avec elles, des capitaux financiers, matériels et humains colossaux. De par leur dimension internationale, les politiques qu’elles appliquent ont une influence sur le commerce à grande échelle et il est dès lors naturel de penser que des analyses et stratégies sans failles sont mises en place pour garantir pérennité à chacune d’entre elles.

Cependant les faits nous démontrent que, même si des plans sont toujours élaborés et mis en place, ceux-ci ne se révèlent pas toujours efficaces. A l’instar d’un epitéen face à son sujet de 42sh, les conseils d'administration ne peuvent qu’espérer adopter la meilleure stratégie, celle qui à leurs yeux semble être la plus appropriée, pour tendre vers le meilleur résultat. Il reste toujours une part d’incertitudes, résidus de points qui auront été négligés, mal évalués ou tout simplement non anticipés.

L’adoption d’une stratégie n’est pas sans conséquences et même de grands comptes, solides en tous points, peuvent à la longue se faire dépasser par un concurrent ayant comme seule arme, une méthode différente. Nous allons d’ailleurs illustrer ce point dans l’exemple qui suit.

\subsection{Netflix vs Blockbuster}
Dans les années 1990, Blockbuster était, par sa valorisation à 8,4 milliards de dollars, le leader incontesté dans le \gls{gls-marche} qui était alors le sien, la location de films. Les clients se rendaient en magasin, louaient une copie au format K7 du film souhaité, et pouvaient en profiter à la maison tout en s’affranchissant des frais qu’aurait constitué l’achat de l'œuvre. Au moment de l’apparition du CD, bien plus fin et permettant un envoi par la poste, un jeune concurrent à Blockbuster eut une idée. Il proposa à ses clients un mode de consommation à distance et par abonnement.

\begin{figure}[b]
    \centering
    \subfloat[\centering Logo Netflix]{{\includegraphics[width=3cm]{netflix.png} }}%
    \qquad
    \subfloat[\centering Logo Blockbuster LLC.]{{\includegraphics[width=3cm]{blockbuster.png} }}%
    \caption{Logos de Netflix et Blockbuster LLC.}%
    \label{fig:netflix-blockbuster-logo}%
\end{figure}

Le fonctionnement était d’une grande simplicité, lorsqu’un CD était retourné par le client, un autre (de son choix) lui était renvoyé, sans qu’il n’ait à se déplacer en magasin. Ce fut le premier grand coup de Netflix. Le succès était au rendez-vous, au point où il formula une proposition de rachat\footnote{\mycitation{netflix-didn-t-kill-blockbuster}} par Blockbuster pour un montant de 50 millions de dollars en 2000, proposition que Blockbuster cependant refusa.

Le second grand coup de Netflix a eu lieu lorsqu’il prit connaissance de l’arrivée d’internet. Conscient de l’attrait de ses clients pour la consommation libre et à distance, il comprit, avant les autres, le potentiel de cette technologie et l’utilisa avec brio pour faire de lui un leader. Sa plateforme de streaming fut lancée en 2007 et enchanta ses consommateurs qui laissèrent tomber en désuétude à la fois le traditionnel déplacement en magasin mais également ses acteurs.

Rappelons bien qu’une entreprise existe car elle répond à un besoin, elle repose sur ce qui est appelé un \textcolor{custom_blue}{\gls{gls-marche}}. Or ce dernier, par son aspect intrinsèquement mobile et évolutif, doit être surveillé et analysé constamment par ceux qui l'exploitent et s’en soustraire revient à mettre en jeu sa pérennité. En effet, si un mode de consommation différent n’est pas perçu à temps, alors l’entreprise peut tout bonnement perdre son \gls{gls-marche} et à terme sa solvabilité.

\definitionbloc{gls-marche}

Le constat est sans équivoque, 23 ans après sa création, Netflix, de par ses 203,7 millions de clients et une valorisation à 203 milliards de dollars en 2020, est devenu le leader de son domaine là où son concurrent d'antan a fini par faire faillite\footnote{Quelques magasins sont encore présents mais la société mère à laquelle ils étaient affiliés n'existe plus} en 2014 \myfigurereference{fig:netflix-vs-blockbuster}.

\begin{figure}
    \centering
    \includegraphics[width=15cm]{netflix-vs-blockbuster.png}
    \caption{Valorisation de Netflix et Blockbuster LLC. \cite{netflix-vs-blockbuster}}
    \label{fig:netflix-vs-blockbuster}
\end{figure}

Par l’écoute attentive de son \gls{gls-marche} et des éléments pouvant le modifier, par une connaissance de ses atouts, et par la réutilisation de ces données pour adopter une stratégie efficace, Netflix a fait preuve de qualité et d’excellence, ce qui lui aura permis d’atteindre les résultats qui sont les siens.

\subsection{Notion de qualité et d'excellence}
La notion de qualité et d’excellence est une qualification prêtée aux entreprises qui suivent une démarche de progrès sur tous domaines. Ayant comme base l’écoute de ses \textcolor{custom_blue}{\gls{gls-parties-prenantes}} et de son \gls{gls-marche}, elles utilisent des processus structurés pour adopter une stratégie d’amélioration et de pérennité. Elles ont conscience de faire partie d’un écosystème fragile et qu’il faut le préserver en mesurant avec précaution l'impact des gouvernances prises.

\definitionbloc{gls-parties-prenantes}

Si un groupe faisant beaucoup de profits, minimise le rôle de ses salariés ou l'environnement, sa stabilité ou sa compétitivité peuvent notablement se dégrader avec le temps, par exemple.

\begin{figure}
    \centering
    \includegraphics[width=6cm]{efqm.png}
    \caption{Logo de l'\acrshort{efqm}}
    \label{fig:efqm-logo}
\end{figure}

Afin d’aider un maximum d’entreprises à devenir des références en matière de qualité et d’excellence et ainsi garantir la vigueur de l’économie européenne, la fondation \textcolor{custom_blue}{\acrshort{efqm}} a été créée par 67 grands dirigeants en 1989. Celle-ci mit en place un guide des bonnes pratiques, le modèle \acrshort{efqm} (sujet de ce rapport) s’adressant à toute entreprise quel que soit sa taille ou son statut\footnote{public/privé/associatif}.

\definitionbloc{gls-efqm}

\subsection{Les objectifs de ce document}
Notez que notre objectif, à travers cet écrit, n’est pas de rentrer dans les aspérités du modèle \acrshort{efqm} ou de faire de vous des experts dans le domaine de la qualité managériale. Bien que nous aborderons la structure du modèle, il nous semble plus judicieux, au contraire, de vous expliquer ses objectifs et la raison qui pousse les entreprises à l’adopter encore aujourd’hui après 30 ans d'existence.

Vous aurez en effet de grandes chances, à travers votre expérience professionnelle d'en entendre à nouveau parler tant son utilisation est répandue. Connaître ses tenants et aboutissants vous permettra ainsi de comprendre la démarche dans laquelle se place votre employeur et quels seront les impacts potentiels à votre égard.

D’autre part, le modèle \acrshort{efqm} délivrant des labels aux entreprises, au moment de choisir une société, que ce soit pour devenir employé ou partenaire, il vous sera bénéfique de savoir quels sont les engagements pris par une société estampillée de qualité et d’excellence.

Dans la suite de ce document nous commencerons par répondre des objectifs de ce modèle à travers le contexte qui aura fait ressentir son besoin auprès des grands dirigeants. Nous détaillerons ensuite le processus d’élaboration utilisé pour la création du modèle 2020, justifiant de la sorte la pertinence du modèle après 30 ans d’existence. Son schéma et les liens entre les différentes parties vous sera expliqué dans notre troisième partie. Nous finirons par une explication autour de l’outil de diagnostic \acrshort{radar} et par les certifications qui découlent du modèle \acrshort{efqm}. Vous aurez ainsi connaissance des intérêts qu’ont les entreprises à se faire contrôler et quelle est la valeur d’un label d’excellence \acrshort{efqm}.

\addcontentsline{toc}{section}{Introduction (EN)}
\section*{Introduction (EN)}

Today's advanced world has made giant companies emerge, bringing with them colossal financial, material and human capital. Because of their international dimension, the policies they apply have an influence on large-scale trade and it is therefore natural to think that flawless analyses and strategies are put in place to guarantee the sustainability of each of them.

However, the evidence shows that, although plans are always made and put in place, they do not always prove to be effective. Like an EPITA student faced with the 42sh subject, boards of directors can only hope to adopt the best strategy, the one that seems to them to be the most appropriate, to achieve the best result. There is always a certain amount of uncertainty, a residue of points that have been neglected, badly evaluated or simply not anticipated.

The adoption of a strategy is not without consequences and even large accounts, solid in all respects, can in the long run be overtaken by a competitor whose only weapon is a different method. We will illustrate this point in the following example.

\subsection{Netflix v. Blockbuster}
In the 1990s, Blockbuster was the undisputed leader in the then-current movie rental \gls{gls-marche-en}, with a valuation of \$8.4 billion. Customers would go to a store, rent a tape of the movie they wanted, and enjoy it at home without the expense of buying the movie. When the CD appeared, which was much thinner and could be sent by mail, a young competitor to Blockbuster had an idea. He proposed to his customers a mode of consumption at a distance and by subscription.

\begin{figure}[b]
    \centering
    \subfloat[\centering Netflix logo]{{\includegraphics[width=3cm]{netflix.png} }}%
    \qquad
    \subfloat[\centering Blockbuster LLC. logo]{{\includegraphics[width=3cm]{blockbuster.png} }}%
    \caption{Netflix and Blockbuster LLC. logos}%
    \label{fig:netflix-blockbuster-logo-en}%
\end{figure}

The operation was very simple, when a CD was returned by the customer, another one (of his choice) was sent back to him, without him having to go to the store. This was Netflix's first big move. It was so successful that Blockbuster was offered to be bought\footnote{\mycitation{netflix-didn-t-kill-blockbuster}} for \$50 million in 2000, but they refused.

Let's remember that a company exists because it responds to a need, it is based on what is called a \textcolor{custom_blue}{\gls{gls-marche-en}}. However, the latter, by its intrinsically mobile and evolving aspect, must be constantly monitored and analyzed by those who exploit it, and to evade it is to put its durability at stake. Indeed, if a different mode of consumption is not perceived in time, then the company can simply lose its \gls{gls-marche-en} and its solvency.

\definitionblocen{gls-marche-en}

The observation is unequivocal, 23 years after its creation, Netflix by its 203.7 million customers and a valuation of \$203 billion in 2020 has become the leader in its field where its erstwhile competitor ended up going bankrupt\footnote{A few stores are still around but the parent company they were affiliated with no longer exists.} in 2014 \myfigurereference{fig:netflix-vs-blockbuster-en}.

\begin{figure}
    \centering
    \includegraphics[width=15cm]{netflix-vs-blockbuster.png}
    \caption{Valuation of Netflix et Blockbuster LLC. \cite{netflix-vs-blockbuster}}
    \label{fig:netflix-vs-blockbuster-en}
\end{figure}

By listening carefully to its \gls{gls-marche-en} and the elements that can change it, by knowing its assets, and by reusing this data to adopt an effective strategy, Netflix has demonstrated quality and excellence, which has enabled it to achieve the results it has.

\subsection{Notion of quality and excellence}
The notion of quality and excellence is a qualification given to companies that follow a progress approach in all areas. Having as a basis the listening of its \textcolor{custom_blue}{\gls{gls-parties-prenantes-en}} and its \gls{gls-marche-en}, they use structured processes to adopt a strategy of improvement and durability. They are aware that they are part of a fragile ecosystem that needs to be preserved by carefully measuring the impact of the governance taken.

\definitionblocen{gls-parties-prenantes-en}

If a group making a lot of profit downplays the role of its employees or the environment, its stability or competitiveness can significantly deteriorate over time, for example.

\begin{figure}
    \centering
    \includegraphics[width=6cm]{efqm.png}
    \caption{EFQM Logo}
    \label{fig:efqm-logo-en}
\end{figure}

In order to help as many companies as possible to become examples of quality and excellence and thus guarantee the strength of the European economy, the \textcolor{custom_blue}{\acrshort{efqm} foundation} was created by 67 leading executives in 1989 and has set up a guide to good practice, the \acrshort{efqm} model, which will be the subject of this report. This guide is intended for any company, regardless of its size or status \footnote{public/private/associative}.

\definitionbloc{gls-efqm-en}

\subsection{The objectives of this document}
Please note that our objective in writing this is not to go into the details of the \acrshort{efqm} model or to make you experts in the field of managerial quality. Although we will discuss the structure of the model, it seems more appropriate to explain its objectives and the reason why companies are still adopting it after 30 years of existence.

Indeed, you will have great chances, through your professional experience, to hear about it again its use being so widespread. Knowing the ins and outs of the process will allow you to understand the approach your employer is taking and the potential impact on you.

On the other hand, since the \acrshort{efqm} model issues labels to companies, when choosing a company, whether to become an employee or a partner, it will be beneficial for you to know what commitments are made by a company stamped with quality and excellence.

In the rest of this document, we will begin by answering the objectives of this model through the context that will have made its need felt by the major leaders. We will then detail the development process used to create the 2020 model, justifying the relevance of the model after 30 years of existence. Its layout and the links between the different parts will be explained in our third part. We will end with an explanation of the \acrshort{radar} diagnostic tool and the certifications that are derived from the \acrshort{efqm} model. You will then learn about the interests of companies to be audited and the value of an \acrshort{efqm} excellence label.
