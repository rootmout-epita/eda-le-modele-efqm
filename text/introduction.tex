\addcontentsline{toc}{section}{Introduction}
\section*{Introduction}
Le monde évolué que nous connaissons à fait émerger des entreprises géantes qui brassent, avec elles, des capitaux financiers, matériels et humains colossaux. De par leur dimension internationale, les politiques qu’elles appliquent ont une influence sur le commerce à grande échelle et il est dès lors naturel de penser que des analyses et stratégies sans failles sont mises en place pour garantir pérennité à chacune d’entre elles.

Cependant les faits nous démontrent que, même si des plans sont toujours élaborés et mis en place, ceux-ci ne se révèlent pas toujours efficaces. A l’instar d’un epitéen face à son sujet de 42sh, les conseils d'administration ne peuvent qu’espérer adopter la meilleure stratégie, celle qui à leurs yeux semble être la plus appropriée, pour tendre vers le meilleur résultat. Il reste toujours une part d’incertitudes, résidus de points qui auront été négligés, mal évalués ou tout simplement non anticipés.

L’adoption d’une stratégie n’est pas sans conséquence et même de grands comptes, solides en tous points, peuvent à la longue se faire dépasser par un concurrent ayant comme seule arme, une méthode différente. Nous allons d’ailleurs illustrer ce point dans l’exemple qui suit.

\subsection{Netflix vs Blockbuster}
Dans les années 1990, Blockbuster était, par sa valorisation à 8,4 milliards de dollars, le leader incontesté dans le marché qui était alors le sien, la location de films. Les clients se déplaçaient en magasin, louaient une copie au format K7 du film souhaité, et pouvaient en profiter à la maison tout en s’affranchissant des frais qu’aurait constitué l’achat de l'œuvre. Au moment de l’apparition du CD, bien plus fin et permettant un envoi par la poste, un jeune concurrent à Blockbuster eut une idée. Il proposa à ses clients un mode de consommation à distance et par abonnement.

\begin{figure}[b]
    \centering
    \subfloat[\centering Logo Netflix]{{\includegraphics[width=3cm]{netflix.png} }}%
    \qquad
    \subfloat[\centering Logo Blockbuster LLC.]{{\includegraphics[width=3cm]{blockbuster.png} }}%
    \caption{Logos de Netflix et Blockbuster LLC.}%
    \label{fig:netflix-blockbuster-logo}%
\end{figure}

Le fonctionnement était d’une grande simplicité, lorsqu’un CD était retourné par le client, un autre (de son choix) lui était renvoyé, sans qu’il n’ait à se rendre en magasin. Ce fut le premier grand coup de Netflix. Le succès était au rendez-vous, au point où il formula une proposition de rachat\footnote{\mycitation{netflix-didn-t-kill-blockbuster}} par Blockbuster pour un montant de 50 millions de dollars en 2000, proposition que Blockbuster cependant refusa.

Le second grand coup de Netflix a eu lieu lorsqu’il prit connaissance de l’arrivée d’internet. Conscient de l’attrait de ses clients pour la consommation libre et à distance, il comprit, avant les autres, le potentiel de cette technologie et l’utilisa avec brio pour faire de lui un leader. Sa plateforme de streaming fut lancée en 2007 et enchanta ses consommateurs qui laissèrent tomber en désuétude à la fois le traditionnel déplacement en magasin mais également ses acteurs.

Rappelons bien qu’une entreprise existe car elle répond à un besoin, elle repose sur ce qui est appelé un \gls{gls-marche}. Or ce dernier, par son aspect intrinsèquement mobile et évolutif, doit être surveillé et analysé constamment par ceux qui l'exploitent et s’en soustraire revient à mettre en jeu sa pérennité. En effet, si un mode de consommation différent n’est pas perçu à temps, alors l’entreprise peut tout bonnement perdre son marché ainsi que sa solvabilité.

\definitionbloc{gls-marche}

Le constat est sans équivoque, 23 ans après sa création, Netflix par ses 203,7 millions de clients et une valorisation à 203 milliards de dollars en 2020 est devenu le leader de son domaine là où son concurrent d'antan a fini par faire faillite\footnote{Quelques magasins sont encore présents mais la société mère à laquelle ils étaient affiliés n'existe plus} en 2014 \myfigurereference{fig:netflix-vs-blockbuster}.

\begin{figure}
    \centering
    \includegraphics[width=10cm]{netflix-vs-blockbuster.png}
    \caption{Valorisation de Netflix et Blockbuster LLC. \cite{netflix-vs-blockbuster}}
    \label{fig:netflix-vs-blockbuster}
\end{figure}

Par l’écoute attentive de son marché et des éléments pouvant le modifier, par une connaissance de ses atouts, et par la réutilisation de ces données pour adopter une stratégie efficace, Netflix a fait preuve de qualité et d’excellence, ce qui lui aura permis d’atteindre les résultats qui sont les siens.

\subsection{Notion de qualité et d'excellence}
La notion de qualité et d’excellence est une qualification prêtée aux entreprises qui suivent une démarche de progrès sur tous domaines. Ayant comme base l’écoute de ses \gls{gls-parties-prenantes} et de son marché, elles utilisent des processus structurés pour adopter une stratégie d’amélioration et de pérennité. Elles ont conscience de faire partie d’un écosystème fragile et qu’il faut préserver en mesurant avec précaution l'impact des gouvernances prises.

\definitionbloc{gls-parties-prenantes}

Si un groupe faisant beaucoup de profits, minimise le rôle de ses salariés ou l'environnement, sa stabilité ou sa compétitivité peuvent notablement se dégrader avec le temps, par exemple.

\begin{figure}
    \centering
    \includegraphics[width=10cm]{efqm.png}
    \caption{Logo de l'EFQM}
    \label{fig:efqm-logo}
\end{figure}

Afin d’aider un maximum d’entreprises à devenir des exemples de qualité et d’excellence et ainsi garantir la vigueur de l’économie européenne, la fondation EFQM a été créée par 67 grands dirigeants en 1989 et a mis en place un guide des bonnes pratiques, le modèle EFQM, qui sera le sujet de ce rapport. Ce guide s’adresse à toute entreprise quel que soit sa taille ou son statut\footnote{public/privé/associatif}.

\definitionbloc{gls-efqm}

\subsection{Les objectifs de ce document}
Notez que notre objectif, à travers cet écrit, n’est pas de rentrer dans les aspérités du modèle EFQM ou de faire de vous des experts dans le domaine de la qualité managériale. Bien que nous aborderons la structure du modèle, il nous semble plus judicieux, au contraire, de vous expliquer ses objectifs et la raison qui pousse les entreprises à l’adopter encore aujourd’hui après 30 ans d'existence.

Vous aurez en effet de grandes chances, à travers votre expérience professionnelle d'en entendre à nouveau parler tant son utilisation est répandue. Connaître ses tenants et aboutissants vous permettra ainsi de comprendre la démarche dans laquelle se place votre employeur et quels seront les impacts potentiels à votre égard.

D’autre part, le modèle EFQM délivrant des labels aux entreprises, au moment de choisir une société, que ce soit pour devenir employé ou partenaire, il vous sera bénéfique de savoir quels sont les engagements pris par une société estampillée de qualité et d’excellence.

Dans la suite de ce document nous commencerons par répondre des objectifs de ce modèle à travers le contexte qui aura fait ressentir son besoin auprès des grands dirigeants. Nous détaillerons ensuite le processus d’élaboration utilisé pour la création du modèle 2020, justifiant de la sorte la pertinence du modèle après 30 ans d’existence. Son schéma et les liens entre les différentes parties vous sera expliqué dans notre troisième partie. Nous finirons par une explication autour de l’outil de diagnostic RADAR et par les certifications qui découlent du modèle EFQM. Vous aurez ainsi connaissance des intérêts qu’ont les entreprises à se faire contrôler et quelle est la valeur d’un label d’excellence EFQM.

\color{custom_blue}
\noindent\makebox[\linewidth]{\rule{1cm}{2pt}}
\color{black}

\vspace{1cm}

\noindent [RISSON PART]
